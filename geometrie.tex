\section{Geometrie}

%\lstinputlisting{geometrie/old/header.cpp}

%\subsection{Geraden und Punkte}

%\lstinputlisting{geometrie/old/gerade.cpp}

%\subsection{Dreiecke und Liniensegmente}

%\lstinputlisting{geometrie/old/dreieck.cpp}
%Abstand eines Punktes, Abstand eines anderen Segments.

%\subsection{Kreise}

%\lstinputlisting{geometrie/old/kreis.cpp}
%Kreis durch drei Punkte, Kreis durch zwei Punkte mit geg. Radius.

%\subsection{Polygone}

%Die \(N\) Punkte eines Polygons werden mit \((x_i,y_i)\) f"ur 
%\(i=1,\,\ldots,\,N\) bezeichnet. Konvention: \((x_0,y_0)=(x_N,y_N)\).

%Die vorzeichenbehaftete Fl"ache \(A\) berechnet sich wie folgt:
%\[ A = \frac{1}{2} \sum_{i=1}^{N} x_{i-1} y_i - x_i y_{i - 1} \]

%Bei Polygonen mit ganzzahligen Koordinaten gilt f"ur die Fl"ache \(A\),
%die Anzahl der Gitterpunkte auf dem Rand \(B\) und die derjenigen im Innern \(I\):
%\[ I = A - \frac{B}{2} + 1 \]

%\subsubsection{Punkt in Polygon}

%\lstinputlisting{geometrie/old/poly-punkt.cpp}

\subsection{Verschiedenes}

\lstinputlisting{niklas/geometry.cpp}

\subsection{Graham's Scan + max. Abstand}

\lstinputlisting{geometrie/_Modul_Farthestpoints_Graham.cpp}

%\subsection{3D Konvexe H"ulle}

%\lstinputlisting{geometrie/3DConvexHull.cpp}

%\subsection{Kernel of polygon}

%\lstinputlisting{geometrie/Kernel.cpp}

%\subsubsection{Clipping an Halbebene bzw. Gerade}
%\lstinputlisting{geometrie/old/clip.cpp}

% Punkt in konvexem Polygon in O(log n), Clipping an Gerade.  (Delaunay-)Triangulation?

%\subsection{Rasterung von Linien/Kreisen (Bresenham)}
%\lstinputlisting{geometrie/old/bresenham.cpp}

%\subsection{Sph"arischer Abstand zwischen Punkt und Punkt (Gro"skreis)}
